% Options for packages loaded elsewhere
\PassOptionsToPackage{unicode}{hyperref}
\PassOptionsToPackage{hyphens}{url}
\PassOptionsToPackage{dvipsnames,svgnames,x11names}{xcolor}
%
\documentclass[
  11pt,
  oneside]{book}
\usepackage{amsmath,amssymb}
\usepackage{iftex}
\ifPDFTeX
  \usepackage[T1]{fontenc}
  \usepackage[utf8]{inputenc}
  \usepackage{textcomp} % provide euro and other symbols
\else % if luatex or xetex
  \usepackage{unicode-math} % this also loads fontspec
  \defaultfontfeatures{Scale=MatchLowercase}
  \defaultfontfeatures[\rmfamily]{Ligatures=TeX,Scale=1}
\fi
\usepackage{lmodern}
\ifPDFTeX\else
  % xetex/luatex font selection
\fi
% Use upquote if available, for straight quotes in verbatim environments
\IfFileExists{upquote.sty}{\usepackage{upquote}}{}
\IfFileExists{microtype.sty}{% use microtype if available
  \usepackage[]{microtype}
  \UseMicrotypeSet[protrusion]{basicmath} % disable protrusion for tt fonts
}{}
\makeatletter
\@ifundefined{KOMAClassName}{% if non-KOMA class
  \IfFileExists{parskip.sty}{%
    \usepackage{parskip}
  }{% else
    \setlength{\parindent}{0pt}
    \setlength{\parskip}{6pt plus 2pt minus 1pt}}
}{% if KOMA class
  \KOMAoptions{parskip=half}}
\makeatother
\usepackage{xcolor}
\usepackage{color}
\usepackage{fancyvrb}
\newcommand{\VerbBar}{|}
\newcommand{\VERB}{\Verb[commandchars=\\\{\}]}
\DefineVerbatimEnvironment{Highlighting}{Verbatim}{commandchars=\\\{\}}
% Add ',fontsize=\small' for more characters per line
\usepackage{framed}
\definecolor{shadecolor}{RGB}{248,248,248}
\newenvironment{Shaded}{\begin{snugshade}}{\end{snugshade}}
\newcommand{\AlertTok}[1]{\textcolor[rgb]{0.94,0.16,0.16}{#1}}
\newcommand{\AnnotationTok}[1]{\textcolor[rgb]{0.56,0.35,0.01}{\textbf{\textit{#1}}}}
\newcommand{\AttributeTok}[1]{\textcolor[rgb]{0.13,0.29,0.53}{#1}}
\newcommand{\BaseNTok}[1]{\textcolor[rgb]{0.00,0.00,0.81}{#1}}
\newcommand{\BuiltInTok}[1]{#1}
\newcommand{\CharTok}[1]{\textcolor[rgb]{0.31,0.60,0.02}{#1}}
\newcommand{\CommentTok}[1]{\textcolor[rgb]{0.56,0.35,0.01}{\textit{#1}}}
\newcommand{\CommentVarTok}[1]{\textcolor[rgb]{0.56,0.35,0.01}{\textbf{\textit{#1}}}}
\newcommand{\ConstantTok}[1]{\textcolor[rgb]{0.56,0.35,0.01}{#1}}
\newcommand{\ControlFlowTok}[1]{\textcolor[rgb]{0.13,0.29,0.53}{\textbf{#1}}}
\newcommand{\DataTypeTok}[1]{\textcolor[rgb]{0.13,0.29,0.53}{#1}}
\newcommand{\DecValTok}[1]{\textcolor[rgb]{0.00,0.00,0.81}{#1}}
\newcommand{\DocumentationTok}[1]{\textcolor[rgb]{0.56,0.35,0.01}{\textbf{\textit{#1}}}}
\newcommand{\ErrorTok}[1]{\textcolor[rgb]{0.64,0.00,0.00}{\textbf{#1}}}
\newcommand{\ExtensionTok}[1]{#1}
\newcommand{\FloatTok}[1]{\textcolor[rgb]{0.00,0.00,0.81}{#1}}
\newcommand{\FunctionTok}[1]{\textcolor[rgb]{0.13,0.29,0.53}{\textbf{#1}}}
\newcommand{\ImportTok}[1]{#1}
\newcommand{\InformationTok}[1]{\textcolor[rgb]{0.56,0.35,0.01}{\textbf{\textit{#1}}}}
\newcommand{\KeywordTok}[1]{\textcolor[rgb]{0.13,0.29,0.53}{\textbf{#1}}}
\newcommand{\NormalTok}[1]{#1}
\newcommand{\OperatorTok}[1]{\textcolor[rgb]{0.81,0.36,0.00}{\textbf{#1}}}
\newcommand{\OtherTok}[1]{\textcolor[rgb]{0.56,0.35,0.01}{#1}}
\newcommand{\PreprocessorTok}[1]{\textcolor[rgb]{0.56,0.35,0.01}{\textit{#1}}}
\newcommand{\RegionMarkerTok}[1]{#1}
\newcommand{\SpecialCharTok}[1]{\textcolor[rgb]{0.81,0.36,0.00}{\textbf{#1}}}
\newcommand{\SpecialStringTok}[1]{\textcolor[rgb]{0.31,0.60,0.02}{#1}}
\newcommand{\StringTok}[1]{\textcolor[rgb]{0.31,0.60,0.02}{#1}}
\newcommand{\VariableTok}[1]{\textcolor[rgb]{0.00,0.00,0.00}{#1}}
\newcommand{\VerbatimStringTok}[1]{\textcolor[rgb]{0.31,0.60,0.02}{#1}}
\newcommand{\WarningTok}[1]{\textcolor[rgb]{0.56,0.35,0.01}{\textbf{\textit{#1}}}}
\usepackage{longtable,booktabs,array}
\usepackage{calc} % for calculating minipage widths
% Correct order of tables after \paragraph or \subparagraph
\usepackage{etoolbox}
\makeatletter
\patchcmd\longtable{\par}{\if@noskipsec\mbox{}\fi\par}{}{}
\makeatother
% Allow footnotes in longtable head/foot
\IfFileExists{footnotehyper.sty}{\usepackage{footnotehyper}}{\usepackage{footnote}}
\makesavenoteenv{longtable}
\setlength{\emergencystretch}{3em} % prevent overfull lines
\providecommand{\tightlist}{%
  \setlength{\itemsep}{0pt}\setlength{\parskip}{0pt}}
\setcounter{secnumdepth}{5}
\usepackage{fontspec}
\usepackage{graphicx}
\usepackage{array}
\usepackage{booktabs}
\usepackage{longtable}
\usepackage{amsmath}
\usepackage{float}

\pagestyle{plain}  % this is used to center page numbers throughout the document

\setmainfont{Calibri} % Replace with any system-installed font
\renewcommand{\chaptername}{Section} % Replace chapter names from 'Chapter' to 'Section'
% Set specific margins:
\usepackage[
  top=0.8in,     % Top margin
  bottom=0.8in,  % Bottom margin
  left=0.75in,   % Left margin
  right=0.75in   % Right margin
]{geometry}
\usepackage{booktabs}
\usepackage{longtable}
\usepackage{array}
\usepackage{multirow}
\usepackage{wrapfig}
\usepackage{float}
\usepackage{colortbl}
\usepackage{pdflscape}
\usepackage{tabu}
\usepackage{threeparttable}
\usepackage{threeparttablex}
\usepackage[normalem]{ulem}
\usepackage{makecell}
\usepackage{xcolor}
\ifLuaTeX
  \usepackage{selnolig}  % disable illegal ligatures
\fi
\usepackage[]{natbib}
\bibliographystyle{apalike}
\usepackage{bookmark}
\IfFileExists{xurl.sty}{\usepackage{xurl}}{} % add URL line breaks if available
\urlstyle{same}
\hypersetup{
  colorlinks=true,
  linkcolor={Maroon},
  filecolor={Maroon},
  citecolor={Blue},
  urlcolor={Blue},
  pdfcreator={LaTeX via pandoc}}

\author{}
\date{\vspace{-2.5em}}

\begin{document}

%\cleardoublepage\newpage\thispagestyle{empty}\null
%\cleardoublepage\newpage
\thispagestyle{empty}

\begin{flushright}
EPA-454/R-23-001a \\
March 2023
\end{flushright}

\includegraphics[width=0.20\textwidth]{figures/epa-logo.jpg}

\begin{center}
\textbf{\Large{2023 National Emissions Inventory Technical Support Document}}
\end{center}

\vspace{0.60\textheight}

\begin{center}
U.S. Environmental Protection Agency \\
Office of Air Quality Planning and Standards \\
Air Quality Assessment Division \\
Research Triangle Park, NC
\end{center}

%\setlength{\abovedisplayskip}{-5pt}
%\setlength{\abovedisplayshortskip}{-5pt}

{
\hypersetup{linkcolor=}
\setcounter{tocdepth}{1}
\tableofcontents
}
\listoffigures
\listoftables
\chapter*{Acronyms and Chemical Notations}\label{acronyms-and-chemical-notations}
\addcontentsline{toc}{chapter}{Acronyms and Chemical Notations}

\begin{longtable}{ll}
\toprule
AEDT & Aviation Environmental Design Tool\\
AERR & Air Emissions Reporting Rule\\
APU & Auxiliary power unit\\
BEIS & Biogenics Emissions Inventory System\\
C1 & Category 1 (commercial marine vessels)\\
\addlinespace
C2 & Category 2 (commercial marine vessels)\\
C3 & Category 3 (commercial marine vessels)\\
CAMD & Clean Air Markets Division (of EPA Office of Air and Radiation)\\
CAP & Criteria Air Pollutant\\
CBM & Coal bed methane\\
\addlinespace
CDL & Cropland Data Layer\\
CEC & North American Commission for Environmental Cooperation\\
CEM & Continuous Emissions Monitoring\\
CERR & Consolidated Emissions Reporting Rule\\
CFR & Code of Federal Regulations\\
\addlinespace
CH4 & Methane\\
CMU & Carnegie Mellon University\\
CMV & Commercial marine vessels\\
CNG & Compressed natural gas\\
CO & Carbon monoxide\\
\addlinespace
CO2 & Carbon dioxide\\
CSV & Comma Separated Variable\\
E10 & 10\% ethanol gasoline\\
EDMS & Emissions and Dispersion Modeling System\\
EF & emission factor\\
\addlinespace
EGU & Electric Generating Utility\\
EIS & Emission Inventory System\\
EAF & Electric arc furnace\\
EF & Emission factor\\
EI & Emissions Inventory\\
\addlinespace
EIA & Energy Information Administration\\
EMFAC & Emission FACtor (model) – for California\\
EPA & Environmental Protection Agency\\
ERG & Eastern Research Group\\
ERTAC & Eastern Regional Technical Advisory Committee\\
\addlinespace
FAA & Federal Aviation Administration\\
FACTS & Forest Service Activity Tracking System\\
FCCS & Fuel Characteristic Classification System\\
FETS & Fire Emissions Tracking System\\
FWS & United States Fish and Wildlife Service\\
\addlinespace
FRS & Facility Registry System\\
GHG & Greenhouse gas\\
GIS & Geographic information systems\\
GSE & Ground support equipment\\
HAP & Hazardous Air Pollutant\\
\addlinespace
HCl & Hydrogen chloride (hydrochloric acid)\\
Hg & Mercury\\
HMS & Hazard Mapping System\\
ICR & Information collection request\\
I/M & Inspection and maintenance\\
\addlinespace
IPCC & Intergovernmental Panel on Climate Change\\
IPM & Integrated Planning Model\\
LRTAP & Long-range Transboundary Air Pollution\\
LTO & Landing and takeoff\\
LPG & Liquified Petroleum Gas\\
\addlinespace
MARAMA & Mid-Atlantic Regional Air Management Association\\
MATS & Mercury and Air Toxics Standards\\
MCIP & Meteorology-Chemistry Interface Processor\\
MMT & Manure management train\\
MOBILE6 & Mobile Source Emission Factor Model, version 6\\
\addlinespace
MODIS & Moderate Resolution Imaging Spectroradiometer\\
MOVES & Motor Vehicle Emissions Simulator\\
MW & Megawatts\\
MWC & Municipal waste combustors\\
NAA & Nonattainment area\\
\addlinespace
NAAQS & National Ambient Air Quality Standards\\
NAICS & North American Industry Classification System\\
NASS & USDA National Agriculture Statistical Service\\
NATA & National Air Toxics Assessment\\
NCD & National County Database\\
\addlinespace
NEEDS & National Electric Energy Data System (database)\\
NEI & National Emissions Inventory\\
NESCAUM & Northeast States for Coordinated Air Use Management\\
NFEI & National Fire Emissions Inventory\\
NG & Natural gas\\
\addlinespace
NH3 & Ammonia\\
NMIM & National Mobile Inventory Model\\
NO & Nitrous oxide\\
NO2 & Nitrogen dioxide\\
NOAA & National Oceanic and Atmospheric Administration\\
\addlinespace
NOx & Nitrogen oxides\\
O3 & Ozone\\
OAQPS & Office of Air Quality Standards and Planning (of EPA)\\
OEI & Office of Environmental Information (of EPA)\\
ORIS & Office of Regulatory Information Systems\\
\addlinespace
OTAQ & Office of Transportation and Air Quality (of EPA)\\
PADD & Petroleum Administration for Defense Districts\\
PAH & Polycyclic aromatic hydrocarbons\\
Pb & Lead\\
PCB & Polychlorinated biphenyl\\
\addlinespace
PFAS & Per- and polyfluoroalkyl substances\\
PM & Particulate matter\\
PM25-CON & Condensable PM2.5\\
PM25-FIL & Filterable PM2.5\\
PM25-PRI & Primary PM2.5 (condensable plus filterable)\\
\addlinespace
PM2.5 & Particulate matter 2.5 microns or less in diameter, synonymous with PM25-PRI\\
PM10 & Particular matter 10 microns or less in diameter, synonymous with PM10-PRI\\
PM10-FIL & Filterable PM10\\
PM10-PRI & Primary PM10 (condensable plus filterable)\\
POM & Polycyclic organic matter\\
\addlinespace
POTW & Publicly Owned Treatment Works\\
PSC & Program system code (in EIS)\\
RFG & Reformulated gasoline\\
RPD & Rate per distance\\
RPP & Rate per profile\\
\addlinespace
RPV & Rate per vehicle\\
RVP & Reid Vapor Pressure\\
Rx & Prescribed (fire)\\
SCC & Source classification code\\
SEDS & State Energy Data System\\
\addlinespace
SFv1 & SMARTFIRE version 1\\
SFv2 & SMARTFIRE version 2\\
S/L/T & State, local, and tribal (agencies)\\
SMARTFIRE & Satellite Mapping Automated Reanalysis Tool for Fire Incident Reconciliation\\
SMOKE & Sparse Matrix Operator Kernel Emissions\\
\addlinespace
SO2 & Sulfur dioxide\\
SO4 & Sulfate\\
TAF & Terminal Area Forecasts\\
TRI & Toxics Release Inventory\\
UNEP & United Nations Environment Programme\\
\addlinespace
UNFCCC & United Nations Framework Convention on Climate Change\\
USDA & United States Department of Agriculture\\
VMT & Vehicle miles traveled\\
VOC & Volatile organic compounds\\
USFS & United States Forest Service\\
\addlinespace
WebFIRE & Factor Information Retrieval System\\
WLF & Wildland fire\\
WRAP & Western Regional Air Partnership\\
WRF & Weather Research and Forecasting Model\\
\bottomrule
\end{longtable}

\chapter{Cooking}\label{cooking}

\section{Sector Descriptions and Overview}\label{sector-descriptions-and-overview}

Cooking is a source of both gaseous and particulate pollutants {[}\hyperref[cooking-references]{ref 1} -- \hyperref[cooking-references]{ref 7}{]}. Primary particulate emissions from cooking are predominantly organic carbon {[}\hyperref[cooking-references]{ref 2}, \hyperref[cooking-references]{ref 5}{]} and positive matrix factorization analysis of Aerosol Mass Spectrometry data from multiple cities in the United States suggest cooking contributes \textasciitilde16\% of observed organic aerosol and \textasciitilde8\% of observed PM2.5 in urban areas {[}\hyperref[cooking-references]{ref 8} -- \hyperref[cooking-references]{ref 9}{]}. In the NEI, emissions are estimated from the cooking of meat, including steak, hamburger, poultry, pork, and seafood, and french fries on five different cooking devices: chain-driven (conveyorized) charbroilers, underfired charbroilers, deep-fat fryers, flat griddles and clamshell griddles. The table below notes all SCCs covered in this source category and the SCCs for which the EPA generates default emissions.

Table \ref{tab:cooking-sccs} notes all SCCs covered in this source category and the SCCs for which the EPA generates default emissions.

\begin{table}
\centering
\caption{\label{tab:cooking-sccs}Cooking SCCs in the 2023 NEI.}
\centering
\resizebox{\ifdim\width>\linewidth\linewidth\else\width\fi}{!}{
\begin{tabular}[t]{rlllll}
\toprule
SCC & SCC Level 1 & SCC Level 2 & SCC Level 3 & SCC Level 4 & EPA\\
\midrule
2302002000 & Industrial Processes & Food and Kindred Products: SIC 20 & Commercial Cooking - Charbroiling & Charbroiling Total & \\
2302002100 & Industrial Processes & Food and Kindred Products: SIC 20 & Commercial Cooking – Charbroiling & Conveyorized Charbroiling & X\\
2302002200 & Industrial Processes & Food and Kindred Products: SIC 20 & Commercial Cooking – Charbroiling & Under-fired Charbroiling & X\\
2302003000 & Industrial Processes & Food and Kindred Products: SIC 20 & Commercial Cooking - Frying & Deep Fat Frying & X\\
2302003100 & Industrial Processes & Food and Kindred Products: SIC 20 & Commercial Cooking – Frying & Flat Griddle Frying & X\\
\addlinespace
2302003200 & Industrial Processes & Food and Kindred Products: SIC 20 & Commercial Cooking – Frying & Clamshell Griddle Frying & X\\
2812001000 & Miscellaneous Area Sources & Food and Kindred Products: SIC 20 & Residential Cooking - Total & Total & X\\
2810025000 & Miscellaneous Area Sources & Other Combustion & Residential Grilling & Total & \\
\bottomrule
\end{tabular}}
\end{table}

\section{EPA-developed estimates}\label{epa-developed-estimates}

To estimate emissions from commercial establishments, the amount of meat and french fries cooked on various cooking devices in each county is estimated. These estimates are year-specific and based on activity statistics from a 2001 telephone survey in southern California \hyperref[cooking-references]{{[}ref 10{]}} and annually varying restaurant count statistics from Open Street Map \hyperref[cooking-references]{{[}ref 11{]}}. These meat consumption estimates are then scaled to match estimates of commercially consumed meat consumption statistics from the U.S. Department of Agriculture's Food Availability Data System \hyperref[cooking-references]{{[}ref 12{]}}. The amount of french fries cooked by the foodservice industry is from a report prepared for Potatoes USA \hyperref[cooking-references]{{[}ref 13{]}}. For residential cooking sources, the mass of residentially consumed meat consumption statistics from the U.S. Department of Agriculture's Food Availability Data System are also used. The total amount of meat or french fries cooked on each device is then multiplied by emissions factors for relevant pollutants to estimate emissions from cooking.

\subsection{Activity Data}\label{activity-data}

Activity data is the amount of meat consumed away from home and at home. For commercial cooking, these estimates are based on a 2001 telephone survey in southern California \hyperref[cooking-references]{{[}ref 10{]}} that reports the fraction of restaurants in a county that use different pieces of commercial cooking equipment by restaurant-type, the average number of cooking devices per restaurant, and the average mass of meat cooked on each device. The county-level number of various restaurant-types are retrieved from Open Street Map \hyperref[cooking-references]{{[}ref 11{]}}. Restaurant count statistics are pulled from this database using the SIC codes listed in Table \ref{tab:sic-codes}.

\begin{table}
\centering
\caption{\label{tab:sic-codes}SIC codes used to classify restaurant types.}
\centering
\resizebox{\ifdim\width>\linewidth\linewidth\else\width\fi}{!}{
\begin{tabular}[t]{ll}
\toprule
Restaurant Type & Primary SIC Code\\
\midrule
Ethnic Food & 5812-01\\
Fast Food & 5812-03\\
Family & 5812-05\\
Seafood & 5812-07\\
Steak \& BBQ & 5812-08\\
\bottomrule
\end{tabular}}
\end{table}

County-level restaurant count statistics are then multiplied by the fraction of restaurants in a county that use different pieces of commercial cooking equipment by restaurant-type, the average number of cooking devices per restaurant, and the average mass of meat cooked on each device. These values are summarized in Tables \ref{tab:fraction-appliance} - Tables \ref{tab:mass-appliance}.

\begin{table}[H]
\centering
\caption{\label{tab:fraction-appliance}Fraction of restaurants in a county that use different pieces of commercial cooking equipment by restaurant-type.}
\centering
\resizebox{\ifdim\width>\linewidth\linewidth\else\width\fi}{!}{
\begin{tabular}[t]{lrrrrr}
\toprule
Restaurant Type & Conveyorized Char-broilers & Underfired Char-broilers & Deep-Fat Fryers & Flat Griddles & Clamshell Griddles\\
\midrule
Ethnic & 0.035 & 0.475 & 0.819 & 0.627 & 0.040\\
Fast Food & 0.186 & 0.308 & 0.968 & 0.519 & 0.147\\
Family & 0.101 & 0.609 & 0.914 & 0.829 & 0.014\\
Seafood & 0.000 & 0.526 & 1.000 & 0.368 & 0.105\\
Steak \& BBQ & 0.069 & 0.552 & 0.828 & 0.897 & 0.000\\
\bottomrule
\end{tabular}}
\end{table}

\begin{table}[H]
\centering
\caption{\label{tab:number-appliance}Average number of cooking devices per restaurant. Average of non-specific charbroiler and griddle used, where necessary.}
\centering
\resizebox{\ifdim\width>\linewidth\linewidth\else\width\fi}{!}{
\begin{tabular}[t]{lrrrrr}
\toprule
Restaurant Type & Conveyorized Char-broilers & Underfired Char-broilers & Deep-Fat Fryers & Flat Griddles & Clamshell Griddles\\
\midrule
Ethnic & 1.62 & 1.54 & 1.63 & 1.88 & 1.80\\
Fast Food & 1.07 & 1.58 & 3.10 & 1.43 & 2.09\\
Family & 1.71 & 1.29 & 2.34 & 2.03 & 2.03\\
Seafood & 1.10 & 1.10 & 2.47 & 1.11 & 1.50\\
Steak \& BBQ & 1.56 & 1.63 & 2.42 & 1.35 & 1.35\\
\bottomrule
\end{tabular}}
\end{table}

\begin{table}[H]
\centering
\caption{\label{tab:mass-appliance}Average mass of meat cooked per year on each cooking device (tons).}
\centering
\resizebox{\ifdim\width>\linewidth\linewidth\else\width\fi}{!}{
\begin{tabular}[t]{lrrrrr}
\toprule
Meat Type & Conveyorized Char-broilers & Underfired Char-broilers & Deep-Fat Fryers & Flat Griddles & Clamshell Griddles\\
\midrule
Steak & 6.1 & 4.7 & 4.7 & 4.3 & 2.4\\
Hamburger & 20.7 & 7.0 & 7.1 & 9.4 & 34.2\\
Poultry & 10.7 & 8.4 & 14.9 & 5.2 & 5.7\\
Pork & 1.5 & 3.8 & 1.5 & 2.9 & 3.1\\
Seafood & 3.1 & 3.7 & 4.1 & 2.4 & 16.4\\
\addlinespace
Other & NA & 1.1 & 7.1 & 1.5 & NA\\
\bottomrule
\end{tabular}}
\end{table}

The total mass of each type of meat commercially cooked on each appliance-type in a county is then estimated using the following equation:

\begin{equation} 
  M{c,t,a} = \sum_{r}(R_{c,a} \times A_{r,a} \times N_{r,a}) \times T_{t,a} \times SF_{t}
  \label{eq:mass-meat}
\end{equation}

Where: \newline
\(M_{c,t,a}\) = Estimated mass of commercially cooked meat in county (\emph{c}) for each meat-type (\emph{t}) on each appliance (\emph{a}), in tons \newline
\(R_{c,r}\) = Number of restaurants in county (\emph{c}) for each restaurant-type (\emph{r}), in \# \newline
\(A_{r,a}\) = Estimated percent of restaurants for each restaurant-type (\emph{r}) containing \textgreater{} 1 appliance (\emph{a}), in \% \newline
\(N_{r,a}\) = Estimated number of appliances (\emph{a}) for each restaurant-type (\emph{r}), when appliance (\emph{a}) is present, in \# \newline
\(T_{t,a}\) = Average mass of meat cooked for each meat-type (\emph{t}) on each appliance (\emph{a}), in tons/appliance \newline
\(SF_{t}\) = Scale factor for each meat-type (\emph{t}) applied to ensure national-level meat consumption matches USDA statistics, in \% \newline
\(c\) = County \newline
\(t\) = Meat-Type (e.g., Steak, Hamburger, etc.) \newline
\(r\) = Restaurant-Type (e.g., Fast Food, Family, Seafood, etc.) \newline
\(a\) = Appliance (e.g., Conveyorized Charbroiler, Underfired Charbroiler, etc.) \newline

The Scale Factor in Eqn. \eqref{eq:mass-meat} is year-specific and derived using meat consumption statistics from the U.S. Department of Agriculture's Food Availability Data System \hyperref[cooking-references]{{[}ref 12{]}}. Data on the total retail consumption (disappearance) of beef, pork, veal, and lamb, chicken and turkey, and seafood are retrieved, and using data from the USDA's Food Consumption and Nutrition Intakes, it is assumed that \textasciitilde65\% of meat is consumed at home and \textasciitilde35\% is consumed away from home (restaurants, fast food, school, etc.). These percentages are subsequently used to estimate consumption at home and scale commercial cooking meat consumption. In Table \ref{tab:USDA-consumption}, total residential and commercial meat consumption from the USDA's Food Availability Data System by meat-type estimates are provided for recent years. Table \ref{tab:scale-factor} illustrates the derivation of the Scale Factor used in Eqn. \eqref{eq:mass-meat} for 2021.

\begin{Shaded}
\begin{Highlighting}[]
\FunctionTok{print}\NormalTok{(df)}
\end{Highlighting}
\end{Shaded}

\begin{verbatim}
## # A tibble: 5 x 10
##    Year `U.S. Population [millions]` `Beef [short tons]` `Veal [short tons]`
##   <dbl>                        <dbl>               <dbl>               <dbl>
## 1  2017                         325.             9241066               32433
## 2  2018                         327.             9334729               36908
## 3  2019                         328.             9513368               33726
## 4  2020                         330.             9620124               27430
## 5  2021                         332.             9761741               26559
## # i 6 more variables: `Pork [short tons]` <dbl>, `Lamb [short tons]` <dbl>,
## #   `Broilers [short tons]` <dbl>, `Other Chicken [short tons]` <dbl>,
## #   `Turkey [short tons]` <dbl>, `Fish and Shellfish [short tons]` <dbl>
\end{verbatim}

\begin{table}[H]
\centering
\caption{\label{tab:USDA-consumption}Total Retail Consumption from USDA Food Availability Data System}
\centering
\resizebox{\ifdim\width>\linewidth\linewidth\else\width\fi}{!}{
\begin{tabu} to \linewidth {>{\raggedleft}X>{\raggedleft}X>{\raggedleft}X>{\raggedleft}X>{\raggedleft}X>{\raggedleft}X>{\raggedleft}X>{\raggedleft}X>{\raggedleft}X>{\raggedleft}X}
\toprule
Year & U.S. Population [millions] & Beef [short tons] & Veal [short tons] & Pork [short tons] & Lamb [short tons] & Broilers [short tons] & Other Chicken [short tons] & Turkey [short tons] & Fish and Shellfish [short tons]\\
\midrule
2017 & 325.206 & 9241066 & 32433 & 8084865 & 175957 & 14702927 & 190839 & 2670584 & 1957000\\
2018 & 326.924 & 9334729 & 36908 & 8254606 & 184520 & 15023743 & 210624 & 2639784 & 2006500\\
2019 & 328.476 & 9513368 & 33726 & 8531797 & 187058 & 15529076 & 206207 & 2624113 & 0\\
2020 & 330.114 & 9620124 & 27430 & 8498334 & 200439 & 15794921 & 220422 & 2593854 & 0\\
2021 & 332.141 & 9761741 & 26559 & 8396295 & 224643 & 15940560 & 228949 & 2536547 & 0\\
\bottomrule
\end{tabu}}
\end{table}

\noindent
* = Consumption statistics for fresh and frozen fish and shellfish is not available after 2018

\begin{table}[H]
\centering
\caption{\label{tab:scale-factor}Scale Factor Derivation}
\centering
\resizebox{\ifdim\width>\linewidth\linewidth\else\width\fi}{!}{
\begin{tabular}[t]{lllllll}
\toprule
Quantity & Steak\textbackslash{}*\textbackslash{}* & Hamburger\textbackslash{}*\textbackslash{}* & Poultry & Pork & Seafood & Other\\
\midrule
USDA Consumption at Home\textbackslash{}* & 3,172,566 & 3,172,566 & 12,158,936 & 5,457,592 & 1,305,599 & 163,281\\
USDA Commercial Consumption\textbackslash{}* & 1,708,305 & 1,708,305 & 6,547,120 & 2,938,703 & 700,901 & 87,921\\
Base Methods Commercial Consumption & 8,583,805 & 18,206,818 & 22,104,545 & 3,986,449 & 8,392,852 & 8,888,702\\
Scale Factor & 0.20 & 0.09 & 0.30 & 0.74 & 0.08 & 0.01\\
\bottomrule
\end{tabular}}
\end{table}

\noindent
* = Using data from the USDA's Food Consumption and Nutrition Intakes, it is assumed that \textasciitilde65\% of meat is consumed at home and \textasciitilde35\% is consumed away from home (restaurants, fast food, school, etc.).
** = It is assumed that half of all beef consumption reported in the USDA's Food Availability Data System is steak and half is hamburger.

The amount of french fries cooked in each county is calculated based on the amount of frozen potatoes used in the foodservice industry. The total amount of french fries cooked is reported at the national level. The process used to distribute the national amount of french fries cooked to the county-level is discussed in the next section.

\subsection{Allocation Procedure}\label{allocation-procedure}

\subsection{Emission Factors}\label{emission-factors}

\subsection{Controls}\label{controls}

There are no controls assumed for this category.

\subsection{Emissions}\label{emissions}

\begin{equation} 
  E_{c,a} = Usage_{c,a} \times \frac{EF_{a}}{2000}
  \label{eq:cooking-emissions}
\end{equation}

Where: \newline
\(E_{c,a}\) = Annual emissions in county \emph{c} for application \emph{a}, in short tons \newline
\(Usage_{c,a}\) = Liquid asphalt usage in county \emph{c} for application \emph{a}, in short tons \newline
\(EF_{a}\) = Emission factor for application \emph{a}, in lb/ton asphalt \newline
\(a\) = Application types include hot-mix, warm-mix, cutback, and emulsified \newline

\subsection{Sample Calculations}\label{sample-calculations}

Table \ref{tab:cooking-sample-calculations} contains sample calculations for VOC emissions from emulsified asphalt (SCC: 2461022000). The values in these equations are demonstrating program logic and are not representative of any specific NEI year or county.

\begin{table}
\centering
\caption{\label{tab:cooking-sample-calculations}Sample Calculations}
\centering
\resizebox{\ifdim\width>\linewidth\linewidth\else\width\fi}{!}{
\begin{tabular}[t]{rlll}
\toprule
Eq. \# & Equation & Values & Result\\
\midrule
1 & $Usage_{s,a} = Usage_{sp,a} \times \frac{HA_{s}}{HA_{sp}}$ & $172 \times \frac{6.5}{19.9}$ & 56 short tons of liquid asphalt usage for emulsified applications\\
2 & Only applicable for hot- and warm-mix & -- & --\\
3 & $Usage_{c,a} = Usage_{s,a} \times \frac{\sum_{r}PVMT_{c}}{\sum_{r}PVMT_{s}}$ & $56 \times \frac{2.38E^9}{5.15E^{10}}$ & 2.58 short tons of liquid asphalt usage for emulsified applications\\
4 & $E_{c,a} = Usage_{c,a} \times \frac{EF_{a}}{2000}$ & $2.58 \times \frac{197.52}{2000}$ & 0.26 short tons of VOC emissions from emulsified asphalt\\
\bottomrule
\end{tabular}}
\end{table}

\subsection{Improvements/Changes in the 2023 NEI}\label{improvementschanges-in-the-2023-nei}

\begin{itemize}
\tightlist
\item
  Sector changed from ``Commercial Cooking'' to ``Cooking'' to include both commercial and residential cooking.
\item
  Scale Factor are applied to the average amount of meat cooked on each cooking device to ensure national-level meat consumption matches USDA statistics.
\item
  Emission Factors were updated
\end{itemize}

\subsection{Puerto Rick and U.S. Virgin Islands}\label{puerto-rick-and-u.s.-virgin-islands}

Insufficient data exists to calculate emissions for the counties in Puerto Rico and the US Virgin Islands. As such, emissions are based on two proxy counties in Florida: 12011 (Broward County) for Puerto Rico and 12087 (Monroe County) for the U.S. Virgin Islands. Per-capita emission factors from Broward County and Monroe County are applied to Puerto Rico and the U.S. Virgin Islands, respectively.

\section{References}\label{cooking-references}

\begin{enumerate}
\def\labelenumi{\arabic{enumi}.}
\tightlist
\item
  R. Fortmann, P. Kariher and R. Clayton, Indoor Air Quality: Residential Cooking Exposures, ARCADIS Geraghty \& Miller, Inc., Prepared for the State of California Air Resources Board; Contract Num: 97-330, 2001.
\item
  N. Gysel, W. A. Welch, C. L. Chen, P. Dixit, D. R. Cocker, 3rd and G. Karavalakis, Particulate matter emissions and gaseous air toxic pollutants from commercial meat cooking operations, Journal of Environmental Science, 2018, 65, 162-170.
\item
  F. Klein, S. M. Platt, N. J. Farren, A. Detournay, E. A. Bruns, C. Bozzetti, K. R. Daellenbach, D. Kilic, N. K. Kumar, S. M. Pieber, J. G. Slowik, B. Temime-Roussel, N. Marchand, J. F. Hamilton, U. Baltensperger, A. S. Prevot and I. El Haddad, Characterization of Gas-Phase Organics Using Proton Transfer Reaction Time-of-Flight Mass Spectrometry: Cooking Emissions, Environ Sci Technol, 2016, 50, 1243-1250.
\item
  T. Liu, Z. Wang, D. D. Huang, X. Wang and C. K. Chan, Significant Production of Secondary Organic Aerosol from Emissions of Heated Cooking Oils, Environ Sci Tech Let, 2018, 5, 32-37.
\item
  J. D. McDonald, B. Zielinska, E. M. Fujita, J. C. Sagebiel, J. C. Chow and J. G. Watson, Emissions from charbroiling and grilling of chicken and beef, J Air Waste Manag Assoc, 2003, 53, 185-194.
\item
  J. J. Schauer, M. J. Kleeman, G. R. Cass and B. R. T. Simoneit, Measurement of Emissions from Air Pollution Sources. 1. C1 through C29 Organic Compounds from Meat Charbroiling, Environ Sci Technol, 1999, 33, 1566-1577.
\item
  J. J. Schauer, M. J. Kleeman, G. R. Cass and B. R. T. Simoneit, Measurement of Emissions from Air Pollution Sources. 4. C1-C27 Organic Compounds from Cooking with Seed Oils, Environ Sci Technol, 2002, 36, 567-575.
\item
  P. L. Hayes, A. M. Ortega, M. J. Cubison, K. D. Froyd, Y. Zhao, S. S. Cliff, W. W. Hu, D. W. Toohey, J. H. Flynn, B. L. Lefer, N. Grossberg, S. Alvarez, B. Rappenglueck, J. W. Taylor, J. D. Allan, J. S. Holloway, J. B. Gilman, W. C. Kuster, J. A. De Gouw, P. Massoli, X. Zhang, J. Liu, R. J. Weber, A. L. Corrigan, L. M. Russell, G. Isaacman, D. R. Worton, N. M. Kreisberg, A. H. Goldstein, R. Thalman, E. M. Waxman, R. Volkamer, Y. H. Lin, J. D. Surratt, T. E. Kleindienst, J. H. Offenberg, S. Dusanter, S. Griffith, P. S. Stevens, J. Brioude, W. M. Angevine and J. L. Jimenez, Organic aerosol composition and sources in Pasadena, California, during the 2010 CalNex campaign, J Geophys Res-Atmos, 2013, 118, 9233-9257.
\item
  Y. L. Sun, Q. Zhang, J. J. Schwab, K. L. Demerjian, W. N. Chen, M. S. Bae, H. M. Hung, O. Hogrefe, B. Frank, O. V. Rattigan and Y. C. Lin, Characterization of the sources and processes of organic and inorganic aerosols in New York city with a high-resolution time-of-flight aerosol mass apectrometer, Atmos Chem Phys, 2011, 11, 1581-1602.
\item
  M. Potepan, Charbroiling Activity Estimation, Public Research Institute, Prepared for the California Air Resources Board and the California Environmental Protection Agency; Contract Num: 98-721, 2001.
\item
  \href{https://www.openstreetmap.org/}{Open Street Map}, 2023.
\item
  United States Department of Agriculture, \href{https://www.ers.usda.gov/data-products/food-availability-per-capita-data-system/}{Food Availability per Capita Data System}.
\item
  Technomic, 2020. \href{https://potatoesusa.com/research-reports/category/market-insights/}{Domestic Sales and U.S. Potato Utilization Report}. Prepared for Potatoes USA.
\item
  Norbeck, Joseph, 1997. Further Development of Emission Test Methods and Development of Emission Factors for Various Commercial Cooking Operations. Prepared for the South Coast Air Quality Management District.
\item
  C. A. Alves, M. Evtyugina, E. Vicente, A. Vicente, C. Goncalves, A. I. Neto, T. Nunes and N. Kovats, 2022. ``Outdoor charcoal grilling: Particulate and gas-phase emissions, organic speciation and ecotoxicological assessment.'' Atmos Environ. 285.
\end{enumerate}

\chapter{Solvents: Asphalt}\label{asphalt}

\section{Sector Descriptions and Overview}\label{sector-descriptions-and-overview-1}

Liquid asphalt is a petroleum-derived substance used in paving applications, such as the construction of roads, parking lots, driveways, and airport runways, as well as non-paving applications, such as the manufacturing of roofing shingles. While liquid asphalt can be found in natural deposits, most is produced from crude oil. Vacuum distillation separates components of crude oil based on boiling point. Products generated from this process include naptha, gasoline, diesel, and liquid asphalt, the last of which has a boiling point greater than 500 °C. As a result, most volatile, light fractions of organics are separated from liquid asphalt during distillation and prior to use.

In paving applications, liquid asphalt can be applied cold or heated. If applied cold, additional components must be added to lower the viscosity of the material, which allows it to be spread upon a surface (e.g., roadway surface). Cutback asphalt (SCC: 2461021000) is a cold application process that involves mixing the liquid asphalt with petroleum solvents (e.g., naphtha, kerosene, fuel oil, diesel, etc.). Following application, these higher volatility solvents evaporate, leaving the asphalt in place. Due to this increased organic emissions potential, cutback asphalts have grown less common over time and this process now constitutes \textasciitilde1\% of liquid asphalt is use \hyperref[asphalt-references]{{[}ref 1{]}}. Emulsified asphalt (SCC: 2461022000) is a separate cold application process that utilizes water-based solvents and an emulsifying agent. The result is a stable liquid suspension with asphalt globules. Following application, the additives evaporate and leave the asphalt in place. In contrast to cutback asphalts, emulsified asphalts have become more common in recent years and \textasciitilde10\% of liquid asphalt is now used in these applications.

Liquid asphalt can also be applied heated, which both lowers the viscosity of the material and minimizes the need for added solvents. Hot-mix asphalt application (SCC: 2461025100), which is the traditional method for asphalt pavement production, involves combining the liquid asphalt with aggregate at a hot-mix plant and heating the mixture to +150 °C. The mixture is then hauled to the usage site heated, where it is placed, compacted, and ambiently cooled. This process does not require any solvent additions. Warm-mix asphalt application (SCC: 2461025200) is a more recent technology that enables asphalt pavement production to occur at 20 -- 40 °C cooler temperatures than hot-mix asphalt application. Warm-mix asphalt applications at reduced temperatures now constitute \textasciitilde18\% of asphalt paving applications \hyperref[asphalt-references]{{[}ref 2{]}}. These lower production temperatures promote energy savings through reductions in fuel use and lower emissions at the hot-mix plant. To reduce the viscosity of the liquid asphalt in warm-mix applications, water, water-bearing minerals, chemicals, waxes, organic additives, or a combination of technologies must be added \hyperref[asphalt-references]{{[}ref 3{]}}.

While heated applications processes represent most liquid asphalt paving applications, it has historically been assumed that emissions were sparse due to the removal of the more volatile organics during distillation. Recent research has demonstrated that less volatile organic vapors from liquid asphalt do evaporate at temperatures associated with hot-mix application (\textasciitilde140 °C), warm-mix application (120 °C), and post-application, or ``in-use'' temperatures \hyperref[asphalt-references]{{[}ref 4{]}}. Emission during the ``in-use'' period diffuse from the pavement over time following application.

Roofing asphalts include asphalt cements and emulsions used in the manufacturing of asphalt shingles, asphalt sealant, and roof tar. In 2020, manufacturing of these products consumed \textasciitilde1.95 million short tons of asphalt, which is \textasciitilde9\% of all asphalt generated in the United States. These materials are predominantly applied in ambient conditions (i.e., as asphalt shingles). Roof tar, which is applied hot and at temperatures like hot-mix road asphalt (+150 °C), comprises \textasciitilde5\% of roofing asphalt usage. In addition, all forms of roofing asphalt are exposed to both high temperatures and solar radiation throughout the duration of their life cycles, which have the potential of generating enhancements of emissions.

Table \ref{tab:asphalt-sccs} notes all SCCs covered in this source category and the SCCs for which the EPA generates default emissions.

\begin{table}
\centering
\caption{\label{tab:asphalt-sccs}Asphalt SCCs in the 2023 NEI.}
\centering
\resizebox{\ifdim\width>\linewidth\linewidth\else\width\fi}{!}{
\begin{tabular}[t]{rlllll}
\toprule
SCC & SCC Level 1 & SCC Level 2 & SCC Level 3 & SCC Level 4 & EPA\\
\midrule
2461021000 & Solvent Utilization & Misc Non-industrial: Commercial & Cutback Asphalt & Total: All Solvent Types & X\\
2461022000 & Solvent Utilization & Misc Non-industrial: Commercial & Emulsified Asphalt & Total: All Solvent Types & X\\
2461025000 & Solvent Utilization & Miscellaneous Non-industrial: Commercial & Asphalt Paving: Hot and Warm-mix & Hot and Warm-mix Total: All Solvent Types & \\
2461025100 & Solvent Utilization & Miscellaneous Non-industrial: Commercial & Asphalt Paving: Hot and Warm-mix & Hot-mix Total: All Solvent Types & X\\
2461025200 & Solvent Utilization & Miscellaneous Non-industrial: Commercial & Asphalt Paving: Hot and Warm-mix & Warm-mix Total: All Solvent Types & X\\
\addlinespace
2461026000 & Solvent Utilization & Miscellaneous Non-industrial: Commercial & Asphalt Paving: Road Oil & Total: All Solvent Types & \\
2461023000 & Solvent Utilization & Solvent Utilization & Asphalt Roofing & Total: All Solvent Types & X\\
\bottomrule
\end{tabular}}
\end{table}

\section{EPA-developed estimates}\label{epa-developed-estimates-1}

Usage of liquid asphalt at the state-level for each process is calculated and subsequently allocated to the county-level using estimated vehicle miles traveled on paved roads and construction expenditure statistics for roofing processes. Emission factors consider both application and in-use processes. Net county-level emissions are quantified by multiplying the SCC-specific liquid asphalt usage by SCC-specific emission factors. The sources of data, calculation of state-level, SCC-specific usage results, allocation of state-level usage to the county-level, emission factors, and emission estimates are all discussed in subsequent sections.

\subsection{Activity Data}\label{activity-data-1}

Activity data for these sources are the amount of liquid asphalt used in each process. This includes cutback, emulsified, hot mix, warm mix, and roofing. Each year, the Asphalt Institute releases an asphalt usage survey for the Unites States and Canada that reflects usage among their membership \hyperref[asphalt-references]{{[}ref 1{]}}. The Asphalt Institute estimates that their membership captures 90\% of the United States market and the survey includes usage for cutback, emulsified, a summation of heated application processes, and roofing asphalt at the level of Petroleum Administration for Defense Districts (PADD) and sub-PADD levels. PADDs are geographic aggregations of the 50 states and the District of Columbia that were generated during World War II for purposes of administering oil allocation. At the sub-PADD level, usage resolution is provided in aggregations that include up to six states. Following 2020, the Asphalt Institute no longer publicly released this annual survey. As such, additional methods were used to scale the 2020 Asphalt Institute survey to 2023 values, which are further described below.

Separately, state-level, year-specific data on the production of asphalt pavement (i.e., mass of asphalt binder and aggregate) for heated application and the proportions used in warm mix processes at reduced temperatures are available from a National Asphalt Pavement Association (NAPA) annual report \hyperref[asphalt-references]{{[}ref 1{]}}. This report is not used in-lieu of the Asphalt Institute report because it does not include usage of cutback, emulsified, and roofing asphalts. However, the state-level heated application usage and reduced temperature (i.e., warm mix usage) proportions are used to allocate cutback, emulsified, hot mix, and warm mix usage from the Asphalt Institute survey to the state-level. Therefore, it is assumed that the state-level proportions of liquid paving asphalt used in heated applications within a sub-PADD match the state-level proportions of liquid asphalt used in cold application (i.e., cutback and emulsified) within a sub-PADD. Furthermore, since 2023 data is available from the NAPA annual report, the state-level growth or decline in asphalt pavement usage between 2020 and 2023 from these reports are used to scale the 2020 Asphalt Institute survey to generate 2023 estimates of all paving processes. Allocation of activity data for roofing asphalt utilizes separate methods and is described below. The derivation of state-level, per-application usage is as follows:

\begin{equation} 
  Usage_{s,a} = Usage_{sp,a} \times \frac{HA_{s}}{HA_{sp}}
  \label{eq:state-usage}
\end{equation}

Where: \newline
\(Usage_{s,a}\) = Liquid asphalt usage in state \emph{s} for application \emph{a}, in short tons \newline
\(Usage_{sp,a}\) = sub-PADD usage of liquid asphalt in sub-PADD \emph{sp} associated with state \emph{s} for application \emph{a} from the Asphalt Institute survey, in short tons \newline
\(HA_{s}\) = Heated application usage in state \emph{s} from the NAPA survey, in short tons \newline
\(HA_{sp}\) = Heated application usage in sub-PADD \emph{sp} associated with state \emph{s} from the NAPA survey, in short tons \newline
\(a\) = Application types include hot-mix, warm-mix, cutback, and emulsified \newline
\(sp\) = sub-PADD associated with state \emph{s}. sub-PADDs include the 11 districts included in the Asphalt Institute survey \hyperref[asphalt-references]{{[}ref 1{]}} \newline

An additional transformation must be done to split the state-level, heated application usage into hot-mix and warm-mix application, respectively.

\begin{equation}
  Usage_{s,a} = 
  \begin{aligned}
  &\begin{cases} s
    Usage_{s} \times \frac{WMA_{s}}{HA_{s}} & \text{for warm-mix application}\\ 
    Usage_{s} \times \frac{(1 - WMA_{s})}{HA_{s}} & \text{for hot-mix application }
  \end{cases}
  \end{aligned}
  \label{eq:piecewise-usage}
\end{equation}

Where: \newline
\(Usage_{s,a}\) = Liquid asphalt usage in state \emph{s} for application \emph{a}, in short tons \newline
\(Usage_{s}\) = Liquid asphalt usage in state \emph{s} for heated application, as derived in Eqn. \eqref{eq:state-usage}, in short tons \newline
\(WMA_{s}\) = Warm-mix application usage at reduced temperatures in state \emph{s} from the NAPA survey, in short tons \newline
\(HA_{s}\) = Heated application usage in state \emph{s} from the NAPA survey, in short tons. \newline
\(a\) = Application types include hot-mix and warm-mix. State-level summation across application types in Eqn. \eqref{eq:piecewise-usage} yield the state-level application usage for heated application from Eqn. \eqref{eq:state-usage}. \newline

\subsection{Allocation Procedure}\label{allocation-procedure-1}

State-level asphalt usage for paving applications is allocated to the county-level using vehicular miles traveled on paved roads. As such, it is assumed that county-level paving activity is proportional to the estimated vehicular miles traveled on paved roads within each county in a state. Here, proprietary telematics data was used to generate county-specific, VMT totals on various road-types. Road-types with descriptions associated with paved/unpaved roads were separately summed, with the residual (``unknown'') VMT proportioned to the paved/unpaved assortments using the prior totals. Therefore, it is assumed that the ``unknown'' VMT is proportional to the ``known'' paved/unpaved VMT totals.

County-level paved VMT estimates within a state are then paired with the usage estimates from Eqn. \eqref{eq:state-usage} and Eqn. \eqref{eq:piecewise-usage} to yield county-level asphalt usage for each process (hot mix, warm mix, cutback, and emulsified) as follows:

\begin{equation} 
  Usage_{c,a} = Usage_{s,a} \times \frac{\sum_{r}PVMT_{c}}{\sum_{r}PVMT_{s}}
  \label{eq:county-usage}
\end{equation}

Where: \newline
\(Usage_{c,a}\) = Liquid asphalt usage in county \emph{c} for application \emph{a}, in short tons \newline
\(Usage_{s,a}\) = Liquid asphalt usage in state \emph{s} associated with county \emph{c} for application \emph{a}, in short tons \newline
\(PVMT_{c}\) = Estimated paved vehicular miles traveled in county \emph{c}, in miles \newline
\(PVMT_{s}\) = Estimated paved vehicular miles traveled in state \emph{s} associated with county \emph{c}, in miles \newline
\(a\) = Application types include hot-mix, warm-mix, cutback, and emulsified \newline

Activity data for roofing asphalt is retrieved from the Asphalt Institute report at the sub-PADD level and allocated to the state-level using nonresidential \hyperref[asphalt-references]{{[}ref 6{]}} and state/local construction-put-in-place \hyperref[asphalt-references]{{[}ref 7{]}} statistics from the U.S. Census Bureau's Construction Spending datasets. The state-level 2023 and 2020 expenditure values from these datasets are used to scale the 2020 Asphalt Institute roofing asphalt usage numbers to 2023 estimates. Then, state-level roofing asphalt usage is allocated to the county-level using population data from the U.S. Census Bureau \hyperref[asphalt-references]{{[}ref 8{]}} Roofing asphalt usage spans multiple material-types (e.g., asphalt shingles and roof tar) and it is assumed that 5\% of roofing asphalt is used for roof tar, with the remaining 95\% predominantly being used for shingles. These roof tar/roofing shingles proportions are included in the final weighted emissions factors for 2461023000.

\subsection{Emission Factors}\label{emission-factors-1}

Emission factors for all paving processes (hot-mix, warm-mix, cutback, and emulsified) captures emissions that occur during application and in-use. Both application and in-use emission factors for hot-mix and warm-mix asphalt, as well as the in-use emission factors for cutback and emulsified asphalt, were retrieved from Khare et al., 2020 \hyperref[asphalt-references]{{[}ref 4{]}}. Emission factors associated with application for cutback and emulsified asphalt paving were not updated.

During the hot-mix application process, the asphalt pavement (i.e., mixture of liquid asphalt and aggregate) is heated and applied at elevated temperatures (\textasciitilde150 °C). Emissions are highest when sustained heating is initiated and exponentially decline thereafter. Measurements indicate that the exponential function below fits the dynamic change in emissions over a prolonged experiment (\textgreater{} 6 days). However, hot-mix asphalt is not heated for prolonged periods. Here, it is assumed that the application process takes 5 hours and is meant to capture the time between transport, paving, and ambient cooling.

\begin{equation} 
  EF = 7.7 \times \exp^{(-0.016 \times t)} + 16 \times \exp^{(-0.5 \times t)}
  \label{eq:hot-mix-application-ef}
\end{equation}

Where: \newline
\(EF\) = Emission factor of gas-phase organics, in mg per min per kg asphalt \newline
\(t\) = Time, in hours \newline

Integrating Eqn. \eqref{eq:hot-mix-application-ef} over a period of 5 hours yields an emission factor of 4 g/kg asphalt, or 8.04 lb/short ton asphalt (i.e., 0.4\% emissions by weight), and represents the emissions from hot-mix asphalt during the application process. The warm-mix asphalt application process generally occurs at 20 -- 40 °C cooler temperatures than hot-mix asphalt application. Reducing the asphalt temperature from 140 °C to 120 °C reduced the initial pulse of emissions by \textasciitilde46\% (Fig. S5 of \hyperref[asphalt-references]{{[}ref 4{]}}). As such, a warm-mix application emission factor of 2 g/kg asphalt, or 4.32 lb/short ton asphalt (i.e., 0.2\% emissions by weight), is adopted.

Cutback and emulsified application emission factors are developed using compositional information from material safety and data sheets (MSDS) for cutback \hyperref[asphalt-references]{{[}ref 9{]}}) and emulsified \hyperref[asphalt-references]{{[}ref 10{]}}) asphalt. Assuming a volatilization fraction of 95\% for all components yields an emission factor of 813.96 lb/short ton asphalt (407 g/kg asphalt) for cutback applications and 195.51 lb/short ton asphalt (98 g/kg asphalt) for emulsified applications.

In-use emissions follow application and occur under ambient temperatures. Since emissions are strongly influenced by temperature, climatological variation can impact the speed in which emissions occur. Measurements associated with a sustained heating experiment at 60 °C feature an exponential decline and fit the function below (Eqn. \eqref{eq:hot-mix-in-use-ef}). While 60 °C is above ambient conditions for all locations within the United States, measurements show that emissions flatten within a day and remain near-constant for more than 2 additional days. Here, it is assumed that emissions within 72-hours under 60 °C will occur within 1-year under ambient conditions at all locations within the United States. Integrating Eqn. \eqref{eq:hot-mix-in-use-ef} over a period of 72 hours yields an emission factor of 1 g/kg asphalt, or 2.01 lb/short ton asphalt (i.e., 0.1\% emissions by weight), and represents the emissions from all in-use asphalt paving process.

\begin{equation} 
  EF = 0.1 + 3.3 \times \exp^{(-0.35 \times t)}
  \label{eq:hot-mix-in-use-ef}
\end{equation}

Where: \newline
\(EF\) = Emission factor of gas-phase organics, in mg per min per kg asphalt \newline
\(t\) = Time, in hours \newline

Taken together, the VOC emission factors for all asphalt paving process are the summation of emissions associated with application and in-use.

Emission factors for roofing asphalt materials are retrieved from Khare et al., 2020 \hyperref[asphalt-references]{{[}ref 4{]}}. Emissions associated with application for roof tar and in-use emissions from asphalt shingles and roof tar are separately considered and summed for total roofing asphalt emissions. Roof tar is applied at similar temperatures as hot-mix asphalt used in paving applications. Therefore, we assume a similar application emissions factor (4.0 g / kg asphalt; 8.0 lb / ton asphalt). For in-use emissions, we use the emission factors for asphalt shingles and roofing asphalt under ``no sun'' conditions from Khare et al., 2020 (Table S7 of \hyperref[asphalt-references]{{[}ref 4{]}}) and apply the duration of the experiment (46 hours) to the reported emissions factor (5.8 g / kg asphalt; 11.6 lb / ton asphalt for asphalt shingles; 13.0 g / kg asphalt; 26.0 lb / ton asphalt for roof tar). It should be noted that these are likely lower bound emissions as all roofing asphalt products are exposed to significant sunlight, which enhances emissions, and that emissions did not cease by the end of the experiment. For total roofing emissions, the prior emissions factors are weighted with the assumption that 5\% of roofing asphalt is used for roof tar, with the remaining 95\% predominantly being used for shingles (0.40 lb / ton asphalt for application; 12.32 lb / ton asphalt for in-use).

\subsection{Controls}\label{controls-1}

There are no controls assumed for this category.

\subsection{Emissions}\label{emissions-1}

\begin{equation} 
  E_{c,a} = Usage_{c,a} \times \frac{EF_{a}}{2000}
  \label{eq:asphalt-emissions}
\end{equation}

Where: \newline
\(E_{c,a}\) = Annual emissions in county \emph{c} for application \emph{a}, in short tons \newline
\(Usage_{c,a}\) = Liquid asphalt usage in county \emph{c} for application \emph{a}, in short tons \newline
\(EF_{a}\) = Emission factor for application \emph{a}, in lb/ton asphalt \newline
\(a\) = Application types include hot-mix, warm-mix, cutback, and emulsified \newline

\subsection{Sample Calculations}\label{sample-calculations-1}

Table \ref{tab:asphalt-sample-calculations} contains sample calculations for VOC emissions from emulsified asphalt (SCC: 2461022000). The values in these equations are demonstrating program logic and are not representative of any specific NEI year or county.

\begin{table}
\centering
\caption{\label{tab:asphalt-sample-calculations}Sample Calculations}
\centering
\resizebox{\ifdim\width>\linewidth\linewidth\else\width\fi}{!}{
\begin{tabular}[t]{rlll}
\toprule
Eq. \# & Equation & Values & Result\\
\midrule
1 & $Usage_{s,a} = Usage_{sp,a} \times \frac{HA_{s}}{HA_{sp}}$ & $172 \times \frac{6.5}{19.9}$ & 56 short tons of liquid asphalt usage for emulsified applications\\
2 & Only applicable for hot- and warm-mix & -- & --\\
3 & $Usage_{c,a} = Usage_{s,a} \times \frac{\sum_{r}PVMT_{c}}{\sum_{r}PVMT_{s}}$ & $56 \times \frac{2.38E^9}{5.15E^{10}}$ & 2.58 short tons of liquid asphalt usage for emulsified applications\\
4 & $E_{c,a} = Usage_{c,a} \times \frac{EF_{a}}{2000}$ & $2.58 \times \frac{197.52}{2000}$ & 0.26 short tons of VOC emissions from emulsified asphalt\\
\bottomrule
\end{tabular}}
\end{table}

\subsection{Improvements/Changes in the 2023 NEI}\label{improvementschanges-in-the-2023-nei-1}

Small updates to the usage of asphalt cement for paving applications were made for the 2023 NEI. Specifically, since 2023-specific usage was not publicly released by the Asphalt Institute, other methods were required to scale 2020 values to 2023 (these are noted above). Methods for estimating emissions from roofing asphalts are new to the National Emissions Inventory process. Previously, select stakeholders submitted data for 2461023000. Starting with the 2023 NEI, EPA will generate default emissions estimates for this SCC throughout the nation.

\subsection{Puerto Rick and U.S. Virgin Islands}\label{puerto-rick-and-u.s.-virgin-islands-1}

Insufficient data exists to calculate emissions for the counties in Puerto Rico and the US Virgin Islands. As such, emissions are based on two proxy counties in Florida: 12011 (Broward County) for Puerto Rico and 12087 (Monroe County) for the U.S. Virgin Islands. Per-capita emission factors from Broward County and Monroe County are applied to Puerto Rico and the U.S. Virgin Islands, respectively.

\section{References}\label{asphalt-references}

\begin{enumerate}
\def\labelenumi{\arabic{enumi}.}
\tightlist
\item
  2020 Asphalt Usage Survey for the United States and Canada. The Asphalt Institute, Lexington, KY.
\item
  Williams, B.A., J.R. Willis, \& Shacat, J. (2021). Annual Asphalt Pavement Industry Survey on Recycled Materials and Warm-Mix Asphalt Usage: 2020, 11th Annual Survey (IS 138). National Asphalt Pavement Association, Greenbelt, Maryland. \url{doi:10.13140/RG.2.2.14846.46409}.
\item
  U.S. Department of Transportation, Federal Highway Administration, Center for Accelerating Innovation. \href{https://www.fhwa.dot.gov/innovation/everydaycounts/edc-1/wma-faqs.cfm\#hot}{Warm Mix Asphalt FAQs}.
\item
  Khare, P., Machesky, J., Soto, R., He, M., Presto, A.A., Gentner, D.R., Asphalt-related emissions are a major missing nontraditional source of secondary organic aerosol precursors. Science Advances, 6, 35, 2020. \url{doi:10.1126/sciadv.abb9785}.
\item
  U.S. Department of Transportation, Federal Highway Administration, Functional System Length, \href{https://www.fhwa.dot.gov/policyinformation/statistics/2020/hm51.cfm}{Table HM-51 -- Highways Statistics 2020}.
\item
  \href{https://www.census.gov/construction/c30/xlsx/nrstate.xlsx}{Annual Value of Private Nonresidential Construction Put in Place by State}, 2008-2022, U.S. Census Bureau, 2023.
\item
  \href{https://www.census.gov/construction/c30/xlsx/slstate.xlsx}{Annual Value of State and Local Construction Put in Place by State, 2006-2022}, U.S. Census Bureau, 2023.
\item
  \href{https://www.census.gov/programs-surveys/popest/data/tables.html}{County Population Totals: 2020-2022}, U.S. Census Bureau, 2023.
\item
  Cutback Asphalt MSDS
\end{enumerate}

\resizebox{\ifdim\width>\linewidth\linewidth\else\width\fi}{!}{
\begin{tabular}[t]{ll}
\toprule
Product Supplier & MSDS/SDS ID\\
\midrule
Valero & 2013V04\\
Asphalt Emulsion Industries & CUT-SDS-1\\
Martin Asphalt Company & Jan 2007\\
Mohawk Asphalt Emulsions & UN1999\\
Asphalt \& Fuel Supply & 211\\
\addlinespace
Valero & 211\\
Valero & 210\\
\bottomrule
\end{tabular}}

\resizebox{\ifdim\width>\linewidth\linewidth\else\width\fi}{!}{
\begin{tabular}[t]{llr}
\toprule
Pollutant & Avg. \% by Weight & Emission Factor [lb/ton]*\\
\midrule
Naphtha & 40 & 760.00\\
Naphthalene \& PAH*** & 0.58 & 11.02\\
Toluene*** & 0.59 & 11.21\\
Xylene*** & 0.99 & 18.81\\
Benzene*** & 0.19 & 3.61\\
\addlinespace
Ethylbenzene*** & 0.49 & 9.31\\
Hydrogen Sulfide*** & 0.09 & 1.71\\
Total VOC** &  & 813.96\\
\bottomrule
\end{tabular}}

*Assumes 95\% volatilization
**Excludes hydrogen sulfide (not organic)
***Is a Hazardous Air Pollutant

\begin{enumerate}
\def\labelenumi{\arabic{enumi}.}
\setcounter{enumi}{9}
\tightlist
\item
  Emulsified Asphalt MSDS
\end{enumerate}

\resizebox{\ifdim\width>\linewidth\linewidth\else\width\fi}{!}{
\begin{tabular}[t]{ll}
\toprule
Product Supplier & MSDS/SDS ID\\
\midrule
Marathon & 0137MAR019\\
Marathon & 0138MAR019\\
Asphalt Emulsion Industries & EMU-SDS-1\\
U.S. Oil \& Refining Co. & 951\\
\bottomrule
\end{tabular}}

\resizebox{\ifdim\width>\linewidth\linewidth\else\width\fi}{!}{
\begin{tabular}[t]{llr}
\toprule
Pollutant & Avg. \% by Weight & Emission Factor [lb/ton]*\\
\midrule
Naphtha & 10 & 190.00\\
Naphthalene \& PAH*** & 0.29 & 5.51\\
Hydrogen Sulfide*** & 0.09 & 1.71\\
Total VOC** &  & 195.51\\
\bottomrule
\end{tabular}}

*Assumes 95\% volatilization
**Excludes hydrogen sulfide (not organic)
***Is a Hazardous Air Pollutant

\end{document}
